\documentclass[a4paper,12pt]{article}

\usepackage[utf8]{inputenc}
\usepackage[T1]{fontenc}
\usepackage[english,serbianc]{babel}
\usepackage[default]{lato}
\usepackage{amsmath,amssymb}
\usepackage{lmodern}
\usepackage{iftex}
\usepackage{textcomp}
\usepackage{xcolor,soulutf8}
\usepackage{hyperref}
\usepackage{enumitem}
\usepackage{tikz}
\usetikzlibrary{angles,quotes,patterns,arrows.meta}

\tikzset{>={Latex[width=3mm,length=3mm]}} % bigger arrowheads
\setlength{\emergencystretch}{3em} % prevent overfull lines

\author{Ema Spasić}

\begin{document}

\tableofcontents

\newpage
\setlength{\parindent}{2em}
\setlength{\parskip}{1em}


\hypertarget{Увод}{
\section{Увод}\label{Увод}}

Тригонометрија је део математике и геометрије која се бави израчунавањем
елемената троугла, проналажењем зависности у њиховим односима, као и
успостављањем функција углова који их дефинишу.

Првобитно је искључиво израчунавана вредност елемената троугла.

Њен првобитни циљ је данас превазиђен и примена тригонометрије на основу
израчунавања тригонометријских функција, ван сваког посматрања троугла,
учинило је од тригонометрије значајну област математике и геометрије.
Примењује се у различитим областима као што су инжењерство, архитектура,
геодезија, навигација и астрономија.

Особа одговорна за "модерну" тригонометрију био је ренесансни
математичар из Немачке \hl{Јохан Милер фин Кенигсберг} - познатији као
\hl{Регмомонтанус}. Био је први који је тригонометрију третирао као субјекат
по себи. Даљи напредак тригонометрије направио је Никола Коперник.

% ---------------------------------------------
\newpage
\section{Тригонометријска неједначина}

То су неједначине код којих се непозната јавља као аргумент
тригонометријске функције. Решити тригонометријске неједначине значи
наћи све углове који је задовољавају.

Приликом тражења решења ове неједначине можемо најпре решити
одговарајућу једначину, а затим наћи интервале који се у неједначини
траже.

Свака тригонометријска неједначина после примене тригонометријских
формула и трансформација своди се на основну тригонометријске
неједначину.

% ---------------------------------------------

\subsection{Неједначина $\sin(x)>a,a\in\mathbb{R}$}

\begin{enumerate}[label=\alph*)]
\item $a<-1$, сваки број је решење тригонометријске неједначине, зато што је $-1\leq\sin(x)\leq1$
\item $a>1$, неједначина нема решење
\item $-1\leq a\leq-1$, приступамо решавању тригонометријске неједначине
\end{enumerate}

Решавање:

\begin{figure}
\centering
\begin{tikzpicture}
\clip
  (-5.5,5.5) rectangle (5.5,-0.5);
\draw[->]
  (-5.5,0) -- (5,0) node[right] {$x$};
\draw[->]
  (0,-0.5) -- (0,5) node[above] {$y$};
\coordinate (O) at (0,0);
\coordinate (aux2) at (-4,{4*tan(40)});
\coordinate (aux3) at (4,{4*tan(40)});
\coordinate (left) at (-{4*cos(40)},{4*sin(40)});
\coordinate (right) at ({4*cos(40)},{4*sin(40)});
\coordinate (leftUP) at (-{4*cos(40)},5);
\coordinate (rightUP) at ({4*cos(40)},5);
\coordinate (leftArc) at (-{4.5*cos(40)},{4.5*sin(40)});
\coordinate (rightArc) at ({4.5*cos(40)},{4.5*sin(40)});
\draw[->]
  (O) -- (aux3) node[above] {$\arcsin{a}$};
\draw[->]
  (O) -- (aux2);
\draw
  (left) -- (right);
\draw[thick,red!70!black] 
  (O) circle (4);
\draw[red,<->] % arc
	(O) ++ (leftArc) arc(140:40:4.5);
\path[clip] (O) circle (4); % fill
\fill[draw=black, thick, pattern=north west lines]
	(left) -- (leftUP) -- (rightUP) -- (right) -- cycle;
\end{tikzpicture}
\caption{}
\end{figure}

Решење:

\begin{equation}
\begin{split}
\arcsin(a+2k\pi)<x<\pi-\arcsin(a+2k\pi);k\in\mathbb{Z}\\
x\in\bigcup\limits_{k\in\mathbb{Z}}\arcsin(a+2k\pi);\pi-\arcsin(a+2k\pi)	
\end{split}
\end{equation}

% ---------------------------------------------

\subsubsection{Пример 1}

Решити неједначине:

\begin{enumerate}[label=\alph*)]
\item $\sin(x)>-3$
\item $\sin(x)>3$
\item $\sin(x)>\frac{1}{2}$
\end{enumerate}

Решење:

\begin{enumerate}[label=\alph*)]
\item $\sin(x)>-3$, с обзиром да је $-1\le\sin(x)\le1$ следи да је решење било који реалан број, односно  $x\in\mathbb{R}$
\item $\sin(x)>3$, једначина нема решење због $-1\le\sin(x)\le1$
\item $\sin(x)>\frac{1}{2}$
\end{enumerate}


\begin{figure}
\centering
	\begin{tikzpicture}
	\clip
	  (-5.5,5.5) rectangle (5.5,-0.5);
	\draw[->]
	  (-5.5,0) -- (5,0) node[right] {$x$};
	\draw[->]
	  (0,-0.5) -- (0,5) node[above] {$y$};
	\coordinate (O) at (0,0);
	\coordinate (aux2) at (-4,{4*tan(40)});
	\coordinate (aux3) at (4,{4*tan(40)});
	\coordinate (left) at (-{4*cos(40)},{4*sin(40)});
	\coordinate (right) at ({4*cos(40)},{4*sin(40)});
	\coordinate (leftUP) at (-{4*cos(40)},5);
	\coordinate (rightUP) at ({4*cos(40)},5);
	\coordinate (leftArc) at (-{4.5*cos(40)},{4.5*sin(40)});
	\coordinate (rightArc) at ({4.5*cos(40)},{4.5*sin(40)});
	\draw[->]
	  (O) -- (aux3) node[above] {$\arcsin{\frac{1}{2}}=\frac{\pi}{2}$};
	\draw[->]
	  (O) -- (aux2);
	\draw
	  (left) -- (right);
	\draw[thick,red!70!black] 
	  (O) circle (4);
	\draw[red,<->] % arc
		(O) ++ (leftArc) arc(140:40:4.5);
	\path[clip] (O) circle (4); % fill
	\fill[draw=black, thick, pattern=north west lines]
		(left) -- (leftUP) -- (rightUP) -- (right) -- cycle;
	\end{tikzpicture}
\caption{}
\end{figure}

Решење:

\[\frac{\pi}{6}+2k\pi<x<\frac{5\pi}{6}+2k\pi,k\in\mathbb{Z}\]
\centerline{или}
\[x\in\bigcup_{k\in\mathbb{Z}}\frac{\pi}{6}+2k\pi,\frac{5\pi}{6}+2k\pi\]

% ---------------------------------------------

\subsubsection{Пример 2}

Решити неједначину: $2\sin(2x-\frac{\pi}{3})=1$

Решење:

\[2\sin(2x-\frac{\pi}{3})=1\]

ако решавамо по формули

\[\arcsin\left(\frac{1}{2}+2k\pi\right)<2x-\frac{\pi}{3}<\pi-\arcsin\left(\frac{1}{2}+2k\pi\right)\]
\[\frac{\pi}{6}+2k\pi<2x-\frac{\pi}{3}<\pi-\frac{\pi}{6}+2k\pi\]
\[\frac{\pi}{6}+2k\pi<2x-\frac{\pi}{3}<\frac{5\pi}{6}+2k\pi\]
\[\frac{\pi}{3}+\frac{\pi}{6}+2k\pi<2x<\frac{\pi}{3}+\frac{5\pi}{6}+2k\pi\]
\[\frac{\pi}{2}+2k\pi<2x<\frac{7\pi}{6}+2k\pi\]
\[\frac{\pi}{4}+k\pi<x<\frac{7\pi}{12}+k\pi,k\in\mathbb{R}\]
\centerline{или}
\[x\in\bigcup_{k\in\mathbb{Z}}(\frac{\pi}{4}+k\pi,\frac{7\pi}{12}+k\pi)\]

% ---------------------------------------------

\subsection{Неједначина $\sin(x)<a,a\in\mathbb{R}$}

\begin{enumerate}[label=\alph*)]
\item $a<-1$, с обзиром да је $-1\leq\sin(x)\leq1$ у овом случају немамо решење, односно: $x=\O$
\item $a>1$, сваки реалан број је решење, односно $x\in\mathbb{R}$
\item $-1\leq a\leq-1$, приступамо решавању тригонометријске неједначине
\end{enumerate}

Решавање:

\begin{figure}
\centering
	\begin{tikzpicture}
	\clip
	  (-5.5,5.5) rectangle (5.5,-5.5);
	\draw[->]
	  (-5.5,0) -- (5,0) node[right] {$x$};
	\draw[->]
	  (0,-0.5) -- (0,5) node[above] {$y$};
	\coordinate (O) at (0,0);
	\coordinate (aux2) at (-4,{4*tan(40)});
	\coordinate (aux3) at (4,{4*tan(40)});
	\coordinate (left) at (-{4*cos(40)},{4*sin(40)});
	\coordinate (right) at ({4*cos(40)},{4*sin(40)});
	\coordinate (leftDOWN) at (-5,-5);
	\coordinate (rightDOWN) at (5,-5);
	\coordinate (leftArc) at (-{4.5*cos(40)},{4.5*sin(40)});
	\coordinate (rightArc) at ({4.5*cos(40)},{4.5*sin(40)});
	\draw[->]
	  (O) -- (aux3) node[above] {$2\pi+arcsin{a}$};
	\draw[->]
	  (O) -- (aux2) node[above] {$\pi-arcsin{a}$};
	\draw
	  (left) -- (right);
	\draw[thick,red!70!black] 
	  (O) circle (4);
	\draw[red,<->] % arc
		(O) ++ (leftArc) arc(-220:40:4.5);
	\path[clip] (O) circle (4); % fill
	\fill[draw=black, thick, pattern=north west lines]
		(-5,{4*sin(40)}) -- (leftDOWN) -- (rightDOWN) -- (5,{4*sin(40)}) -- cycle;
	\end{tikzpicture}
\caption{}
\end{figure}

\begin{equation}
\begin{split}
\pi-\arcsin(a+2k\pi)<x<2\pi+\arcsin(a+2k\pi);k\in\mathbb{Z}\\
x\in\bigcup\limits_{k\in\mathbb{Z}}(\pi-\arcsin(a+2k\pi);2\pi+\arcsin(a+2k\pi))
\end{split}
\end{equation}


% ---------------------------------------------
\subsubsection{Пример 1}

Решити неједначине:

\begin{enumerate}[label=\alph*)]
\item $\sin(x)<-2$ , једначина нема решење
\item $\sin(x)<2$, решење: $x\in\mathbb{R}$
\item $\sin(x)<-\frac{\sqrt3}{2}$
\end{enumerate}


\[\frac{4\pi}{3}+2x\pi<x<\frac{5\pi}{3}+2k\pi,x\in\mathbb{Z}\]
\[x\in\bigcup_{k\in\mathbb{Z}}(\frac{4\pi}{3}+2k\pi,\frac{5\pi}{3}+2k\pi)\]

ако решавамо по формули

\[\pi-\arcsin\left(-\frac{\sqrt{3}}{2}\right)+2k\pi<x<2\pi+\arcsin\left(-\frac{\sqrt{3}}{2}\right)+2k\pi\]
\[\pi+\frac{\pi}{3}+2k\pi<x<2\pi-\frac{\pi}{3}+2k\pi\]
\[\frac{4\pi}{3}+2k\pi<x<\frac{5\pi}{3}+2k\pi\]
\centerline{или}
\[x\in\bigcup_{k\in\mathbb{Z}}\frac{4\pi}{3}+2k\pi,\frac{5\pi}{3}+2k\pi\]


% ---------------------------------------------
\subsubsection{Пример 2}

Решити неједначину:

\[\sin(x)\cos\left(\frac{\pi}{6}\right)+\cos(x)\sin\left(\frac{\pi}{6}\right)<\frac{1}{2}\]

Решење:

\[\sin(x)\cos\left(\frac{\pi}{6}\right)+\cos(x)\sin\left(\frac{\pi}{6}\right)<\frac{1}{2}\]
\[\sin\left(x+\frac{\pi}{6}\right)<\frac{1}{2}\]
\[\frac{5\pi}{6}+2k\pi<x+\frac{\pi}{6}<\frac{13\pi}{6}+2k\pi\]
\[\frac{5\pi}{6}-\frac{\pi}{6}+2k\pi<x<\frac{13\pi}{6}-\frac{\pi}{6}+2k\pi\]
\[\frac{4\pi}{6}+2k\pi<x<\frac{12\pi}{6}+2k\pi\]
\[\frac{2\pi}{3}+2k\pi<x<2\pi+2k\pi\]
\centerline{или}
\[x\in\bigcup_{k\in\mathbb{Z}}\frac{2\pi}{3}+2k\pi,2\pi+2k\pi\]

Могли смо да решимо користећи формулу XX


% ---------------------------------------------
\subsubsection{Пример 3}

Решити неједначину: $\sin(x)+\cos(x)<\sqrt{2}$

Решење:

\[\sin(x)+\cos(x)<\sqrt{2}\qquad/:\sqrt{2}\]
\[\frac{1}{\sqrt{2}}\sin(x)+\frac{1}{\sqrt{2}}\cos(x)<\frac{\sqrt{2}}{\sqrt{2}}\]
\[\frac{\sqrt{2}}{2}\sin(x)+\frac{\sqrt{2}}{2}\cos(x)<1\]
\[(\frac{\sqrt{2}}{2}=\cos(\frac{\pi}{4});\frac{\sqrt{2}}{2}=\sin(\frac{\pi}{4}))\]
\[\cos(\frac{\pi}{4})\sin(x)+\sin(\frac{\pi}{4})\cos(x)<1\]
\[\sin(x+\frac{\pi}{4})<1\qquad\text{(на основу адиционе теореме)}\]
\[\text{Пошто је} -1\leqslant\sin(x)\leqslant1\Rightarrow \text{решење је} \forall x\in\mathbb{R}\]
\[\text{сем } \sin\left(x+\frac{\pi}{4}\right)=1\text{ односно }x+\frac{\pi}{4}=\frac{\pi}{2}+2k\pi,k\in\mathbb{Z}\]
\centerline{даје решење}
\[\forall x,x\neq\frac{\pi}{4}+2x\pi,k\in\mathbb{Z}\]



% ---------------------------------------------
\subsection{Неједначина $\cos(x)<b,b\in\mathbb{R}$}

\begin{enumerate}[label=\alph*)]
\item $b<-1$, с обзиром да је $-1\leq\cos(x)\leq1$ сваки реалан број је решење ове неједначине
\item $b>1$, неједначина нема решења, зато што је $-1\leq\cos(x)\leq1$
\item $-1\leq b\leq-1$, приступамо решавању тригонометријске неједначине
\end{enumerate}

\[\arccos(b)+2k\pi<x<2\pi-\arccos(b)+2k\pi,k\in\mathbb{Z}\]
\centerline{или}
\[x\in\bigcup_{k\in\mathbb{Z}}\left(\arccos(b)+2k\pi,2\pi-\arccos(b)+2k\pi\right)\]


% ---------------------------------------------
\subsubsection{Пример 1}

Решити неједначине:

\begin{enumerate}[label=\alph*)]
\item $\cos(x)<4$
\item $\cos(x)<-\frac{5}{2}$
\item $\cos(x)<-\frac{\sqrt(3)}{2}$
\end{enumerate}

Решење:

\begin{enumerate}[label=\alph*)]
\item $\cos(x)<4$, пошто је $-1\leqslant\cos(x)\leqslant1$ решење је $\forall x,x\in\mathbb{R}$
\item $\cos(x)<-\frac{5}{2}$, неједначина нема решење
\item $\cos(x)<-\frac{\sqrt(3)}{2}$
\end{enumerate}

\[\frac{2\pi}{3}+2k\pi<x<\frac{4\pi}{3}+2k\pi,k\in\mathbb{R}\]
\centerline{или}
\[x\in\bigcup_{k\in\mathbb{Z}}\frac{2\pi}{3}+2k\pi,\frac{4\pi}{3}+2k\pi\]


% ---------------------------------------------
\subsection{Неједначина $\cos(x)>b,b\in\mathbb{R}$}


\begin{enumerate}[label=\alph*)]
\item $b<-1$, неједначина нема решења, зато што је $-1\leq\cos(x)\leq1$
\item $b>1$, с обзиром да је $-1\leq\cos(x)\leq1$ сваки реалан број је решење ове неједначине
\item $-1\leq b\leq-1$, приступамо решавању тригонометријске неједначине
\end{enumerate}

\[\arccos(b)+2k\pi<x<2\pi-\arccos(b)+2k\pi,k\in\mathbb{Z}\]
\centerline{или}
\[x\in\bigcup_{k\in\mathbb{Z}}\arccos(b)+2k\pi,2\pi-\arccos(b)+2k\pi\]


\subsubsection{Пример 1}

Решити неједначине:

\begin{enumerate}[label=\alph*)]
\item $\cos(x)>3$
\item $\cos(x)>-3$
\item $\cos(x)>\frac{1}{2}$
\end{enumerate}

Решење:

\begin{enumerate}[label=\alph*)]
\item $\cos(x)>3$ , пошто је $-1\leqslant\cos(x)\leqslant1$ неједначина нема решење
\item $\cos(x)>-3$, пошто је $-1\leqslant\cos(x)\leqslant1$ решење је $\forall x,x\in\mathbb{R}$
\item $\cos(x)>\frac{1}{2}$
\end{enumerate}

\[-\frac{\pi}{2}+2k\pi<x<\frac{\pi}{3}+2k\pi,k\in\mathbb{Z}\]
\centerline{или}
\[x\in\bigcup_{k\in\mathbb{Z}}-\frac{\pi}{3}+2k\pi,\frac{\pi}{3}+2k\pi.\]


% ---------------------------------------------
\subsubsection{Пример 2}

Решити неједначину: $2\cos(3x-\frac{\pi}{3})\ge\sqrt{2}$

Решење:

\[2\cos\left(3x-\frac{\pi}{3}\right)\geq\sqrt{2}\qquad/:2\]
\[\cos\left(3x-\frac{\pi}{3}\right)\geqslant\frac{\sqrt{2}}{2}\]
\[-\frac{\pi}{4}+2k\pi\leq3x-\frac{\pi}{3}\leq\frac{\pi}{4}+2k\pi\]
\[-\frac{\pi}{4}+\frac{\pi}{3}+2k\pi\leq3x\leq\frac{\pi}{4}+\frac{\pi}{3}+2k\pi\]
\[\frac{-3\pi+4\pi}{12}+2k\pi\leq3x\leq\frac{3\pi+4\pi}{12}+2k\pi\]
\[\frac{\pi}{12}+2k\pi\leq3x\leq\frac{7\pi}{12}+2k\pi\qquad/:3\]
\[\frac{\pi}{36}+\frac{2k\pi}{3}\leqslant\frac{7\pi}{36}+\frac{2k\pi}{3},k\in\mathbb{Z}\]
\centerline{или}
\[x\in\bigcup_{x\in\mathbb{Z}}\left[\frac{\pi}{36}+\frac{2k\pi}{3},\frac{7\pi}{36}+\frac{2k\pi}{3}\right]\]

% ---------------------------------------------
\subsubsection{Пример 3}

Решити неједначину: $\sin(x)-\cos(x)\ge1$

Решење:

\[\sin(x)-\cos(x)\geqslant1\qquad/:\sqrt{2}\]
\[\frac{1}{\sqrt{2}}\sin(x)-\frac{1}{\sqrt{2}}\cos(x)\geqslant\frac{1}{\sqrt{2}}\qquad/(-1)\]
\[\frac{\sqrt{2}}{2}\cos(x)-\frac{\sqrt{2}}{2}\sin(x)\leq-\frac{\sqrt{2}}{2}\]
\[\cos\left(\frac{\pi}{4}\right)\cos(x)-\sin\left(\frac{\pi}{4}\right)\sin(x)\leqslant-\frac{\sqrt{2}}{2}\]
\[\cos\left(x+\frac{\pi}{4}\right)\leq\frac{-\sqrt{2}}{2}\]
\[\frac{3\pi}{4}+2k\pi\leqslant x+\frac{\pi}{4}\leqslant\frac{5\pi}{4}+2k\pi\]
\[\frac{3\pi}{4}-\frac{\pi}{4}+2k\pi\leqslant x\leqslant\frac{5\pi}{4}-\frac{\pi}{4}+2k\pi\]
\[\frac{2\pi}{4}+2k\pi\leqslant x\leqslant\frac{4\pi}{4}+2k\pi\]
\[\frac{\pi}{2}+2k\pi\leqslant x\leqslant\pi+2k\pi,k\in\mathbb{Z}\]
\centerline{или}
\[x\in\bigcup_{k\in\mathbb{Z}}\left[\frac{\pi}{2}+2k\pi,\pi+2k\pi\right]\]

% ---------------------------------------------
\newpage
\section{Неједначине са $\tg(x)$ i $\ctg(x)$}

Ове неједначине за разлику од оних са $\sin(x)$ и $\cos(x)$ увек имају решења с обзиром да $\tg(x)$ и $\ctg(x)$ узимају вредности из целог скупа $\mathbb{R}$.

% ---------------------------------------------
\subsection{Неједначина $\tg(x)>a,a\in\mathbb{R}$}

Решење:

\[\arctan(a)+k\pi<x<\frac{\pi}{2}+k\pi,k\in\mathbb{Z}\]
\centerline{или}
\[x\in\bigcup_{k\in\mathbb{Z}}\left(\arctg(a)+k\pi,\frac{\pi}{2}+k\pi\right)\]

% ---------------------------------------------
\subsubsection{Primer 1}

Решити неједначину: $3\tg(x)\ge\sqrt3$

Решење:

\[3\tg(x)\geqslant\sqrt{3}\]
\[\tg(x)\geqslant\frac{\sqrt{3}}{3}\]
\[\frac{\pi}{6}+2k\pi\leqslant x\leqslant\frac{\pi}{2}+k\pi\]
\centerline{или}
\[x\in\bigcup_{k\in\mathbb{Z}}\left[\frac{\pi}{6}+k\pi,\frac{\pi}{2}+k\pi\right]\]

% ---------------------------------------------
\subsubsection{Primer 2}

Решити неједначину: $\tg(3x-\frac{\pi}{4})\ge\sqrt3$

Решење:

\[\frac{\pi}{3}+k\pi<3x-\frac{\pi}{4}<\frac{\pi}{2}+k\pi\]
\[\frac{\pi}{3}+\frac{\pi}{4}+k\pi<3\times<\frac{\pi}{2}+\frac{\pi}{4}+k\pi\]
\[\frac{7\pi}{12}+k\pi<3x<\frac{3\pi}{4}+k\pi\qquad/:3\]
\[\frac{7\pi}{36}+\frac{k\pi}{3}<x<\frac{\pi}{4}+\frac{k\pi}{3}\]
\centerline{или}
\[x\in\bigcup_{k\in\mathbb{Z}}\left(\frac{7\pi}{36}+\frac{k\pi}{3},\frac{\pi}{4}+\frac{k\pi}{3}\right)\]

% ---------------------------------------------
\subsection{Неједначина $\tg(x)<a,a\in\mathbb{R}$}

Решење:

\[-\frac{\pi}{2}+k\pi<x<\arctg(a)+k\pi,k\in\mathbb{Z}\]
\centerline{или}
\[x\in\bigcup_{k\in\mathbb{Z}}\left(-\frac{\pi}{2}+k\pi,\arctan(a)+k\pi\right)\]

% ---------------------------------------------
\subsubsection{Primer 1}

Решити неједначину: $\tg(2x+\frac{\pi}{3})\le1$

Решење:

\[-\frac{\pi}{2}+k\pi<2x+\frac{\pi}{3}<\frac{\pi}{4}+k\pi\]
\[-\frac{\pi}{2}-\frac{\pi}{3}+k\pi<2x<\frac{\pi}{4}-\frac{\pi}{3}+k\pi\]
\[-\frac{5\pi}{6}+k\pi<2x<-\frac{\pi}{12}+k\pi\qquad/:2\]
\[-\frac{5\pi}{12}+\frac{k\pi}{2}<x<-\frac{\pi}{24}+\frac{k\pi}{2},k\in\mathbb{Z}\]
или
\[x\in\bigcup_{k\in\mathbb{Z}}\left(-\frac{5\pi}{12}+\frac{k\pi}{2},-\frac{\pi}{24}+\frac{k\pi}{2}\right)\]


% ---------------------------------------------
\subsection{Неједначина $\ctg(x)>a,a\in\mathbb{R}$}

Решење:

\[k\pi<x<\arctg(a)+k\pi,k\in\mathbb{R}\]
\[x\in\bigcup_{k\in\mathbb{Z}}\left(k\pi,\arctg(a)+k\pi\right)\]


% ---------------------------------------------
\subsubsection{Primer 1}

Решити неједначину:

\[\ctg(2x+\frac{\pi}{4})>\frac{\sqrt{3}}{3}\]

Решење:

\[k\pi<2x+\frac{\pi}{4}<\frac{\pi}{3}+k\pi\]
\[-\frac{\pi}{4}+k\pi<2x<\frac{\pi}{3}-\frac{\pi}{4}+k\pi\]
\[-\frac{\pi}{4}+k\pi<2x<\frac{\pi}{12}+k\pi,k\in\mathbb{Z}\]
\[x\in\bigcup_{k\in\mathbb{Z}}\left(-\frac{\pi}{4}+k\pi,\frac{\pi}{12}+k\pi\right)\]

% ---------------------------------------------
\subsection{Неједначина $\ctg(x)<a,a\in\mathbb{R}$}

Решење:

\[\arctg(a)+k\pi<x<\pi+k\pi\]
\[x\in\bigcup_{k\in\mathbb{Z}}(\arctg(a)+k\pi,\pi+k\pi)\]

% ---------------------------------------------
\subsubsection{Primer 1}

Решити неједначину:

\[\ctg\left(3x+\frac{\pi}{6}\right)<-\sqrt{3}\]

Решење:

\[\frac{5\pi}{6}+\pi\pi<3x+\frac{\pi}{6}<\pi+k\pi\]
\[\frac{5\pi}{6}-\frac{\pi}{6}+k\pi<3x<\pi-\frac{\pi}{6}+k\pi\]
\[\frac{4\pi}{6}+k\pi<3x<\frac{5\pi}{6}+k\pi\]
\[\frac{2\pi}{3}+k\pi<3x<\frac{5\pi}{6}+k\pi\qquad/:3\]
\[\frac{2\pi}{9}+\frac{k\pi}{3}<x<\frac{5\pi}{18}+\frac{k\pi}{3}\]


% ---------------------------------------------
\newpage
\section{Различити типови тригонометријских неједначина}

% ---------------------------------------------
\subsection{Неједначине које се тригономтријским трансформацијама своде на основне}

\subsubsection{Пример 1}

Решити неједначину:

\[\cos^{2}(x)+2\sin(x)>1\]

Решење:

\[1-\sin^{2}(x)+2\sin(x)>1\]
\[\sin(x)(2-\sin(x))>0\]
\centerline{због $-1\le\sin(x)\le1$}
\[2-\sin(x)>0\]
\[\sin(x)>0\]
\centerline{даје}
\[0+2k\pi<x<k\pi+2k\pi\]
\centerline{или}
\[x\in\bigcup_{k\in\mathbb{Z}}(2k\pi,\pi+2k\pi)\]

% ---------------------------------------------
\subsubsection{Пример 2}

Решити неједначину:

\[1+\cos(x)\cos(3x)>\sin^{2}(x)\]

Решење:

\[1+\cos(x)\cos(3x)>\sin^{2}(x)\]
\[\cos^{2}(x)+\cos(x)\cos(3x)>0\]
\[\cos(x)(\cos(x)+\cos(3x))>0\]
\[\cos(x)\cdot2\cos\left(\frac{x+3x}{2}\right)\cdot\cos\left(\frac{x-3x}{2}\right)>0\]
\[2\cos(x)\cdot\cos(2x)\cdot\cos(x)>0\]
\[2\cos^{2}(x)\cdot\cos(2x)>0\]
\[\cos(2x)>0\]
\centerline{даје}
\[-\frac{\pi}{2}+2k\pi<2x<\frac{\pi}{2}+2k\pi\qquad/:2\]
\[-\frac{\pi}{4}+k\pi<x<\frac{\pi}{4}+k\pi,k\in\mathbb{Z}\]
\[x\in\bigcup_{k\in\mathbb{Z}}\left(-\frac{\pi}{4}+k\pi,\frac{\pi}{4}+k\pi\right)\]


% ---------------------------------------------
\subsubsection{Пример 3}

Решити неједначину:

\[\frac{\sin(2x)-\cos(2x)+1}{\sin(2x)+\cos(2x)-1}>0\]

Решење:

\[\frac{\sin(2x)+(1-\cos(2x))}{\sin(2x)-(1-\cos(2x))}>0\]
\[\frac{2\sin(x)\cos(x)+2\sin^{2}(x)}{2\sin(x)\cos(x)-2\sin^{2}(x)}>0\]
\[\frac{2\sin(x)(\cos(x)+\sin(x))}{2\sin(x)(\cos(x)-\sin(x))}>0\]
\[\frac{\cos(x)+\sin(x)}{\cos(x)-\sin(x)}>0\]
\[\frac{\sin\left(\frac{\pi}{2}-x\right)+\sin(x)}{\sin\left(\frac{\pi}{2}-x\right)-\sin(x)}>0\]
\[\frac{2\sin\left(\frac{\frac{\pi}{2}-x+x}{2}\right)\cdot\cos\left(\frac{\frac{\pi}{2}-x-x}{2}\right)}{2\cos\left(\frac{\frac{\pi}{2}-x+x}{2}\right)\cdot\sin\left(\frac{\frac{\pi}{2}-x-x}{2}\right)}>0\]
\[\frac{\sin\left(\frac{\pi}{4}\right)\cdot\cos\left(\frac{\pi}{4}-x\right)}{\cos\left(\frac{\pi}{4}\right)\cdot\sin\left(\frac{\pi}{4}-x\right)}>0\]
\[\ctg\left(\frac{\pi}{4}-x\right)>0\]
\[\tg\left(\frac{\pi}{2}-\left(\frac{\pi}{4}-x\right)\right)>0\]
\[\tg\left(\frac{\pi}{4}+x\right)>0\]
\centerline{даје}
\[0+K\pi<\frac{\pi}{4}+x<\frac{\pi}{2}+K\pi\]
\[-\frac{\pi}{4}+k\pi<x<\frac{\pi}{2}-\frac{\pi}{4}+4\pi\]
\[-\frac{\pi}{4}+k\pi<x<\frac{\pi}{4}+k\pi,k\in\mathbb{R}\]
\[x\in\bigcup_{k\in\mathbb{Z}}\left(-\frac{\pi}{4}+k\pi,\frac{\pi}{4}+k\pi\right)\]


% ---------------------------------------------
\subsection{Неједначине које се сменом своде на основне}

\subsubsection{Пример 1}

Решити неједначину:

\[2\cos^{2}(x)-3\cos(x)+1>0\]


Решење:

\[\text{смена:}\cos(x)=t,-1\leqslant t\leqslant1\]
\[2t^{2}-3t+1>0\]
\[t_{1/2}=\frac{3\pm\sqrt{9-8}}{4}\]
\[t_{1/2}=\frac{3\pm1}{4}\Rightarrow t_{1}=1,t_{2}=\frac{1}{2}
\]
\[t<\frac{1}{2}Vt>1\]

\begin{enumerate}[label=\alph*)]
\item $t>1\Rightarrow\cos(x)>1$ , нема решења пошто је $-1\leqslant\cos(x)\leqslant1$
\item $t<\frac{1}{2}=>\cos(x)<\frac{1}{2}$
\end{enumerate}

\[\frac{\pi}{3}+2k\pi<x<\frac{5\pi}{3}+2k\pi,\qquad k\in\mathbb{Z}\]
\[x\in\bigcup_{k\in\mathbb{Z}}\left(\frac{\pi}{3}+2k^{\pi},\frac{5\pi}{3}+2k^{\pi}\right)\]


% ---------------------------------------------
\subsubsection{Пример 2}

Решити неједначину:

\[\sin(x)-3\cos(x)<1\]

Решење:

\[\text{смена:}\tg\left(\frac{x}{2}\right)=t\]
\[\sin(x)=\frac{2\tg\left(\frac{x}{2}\right)}{1+\tg^{2}\left(\frac{x}{2}\right)}\Rightarrow\sin(x)=\frac{2t}{1+t^{2}}\]
\[\cos(x)=\frac{1-\tg^{2}\left(\frac{x}{2}\right)}{1+\tg^{2}\left(\frac{x}{2}\right)}\Rightarrow\cos(x)=\frac{1-t^{2}}{1+t^{2}}\]

Решавање:

\[\sin(x)-3\cos(x)<1\]
\[\frac{2t}{1+t^{2}}-\frac{3\left(1-t^{2}\right)}{1+t^{e}}<1\qquad/1+t^{2}>0\]
\[2t-3+3t^{2}<1+t^{2}\]
\[2t^{2}+2t-4<0\qquad/:2\]
\[t^{2}+t-2<0\]
\[t_{1/2}=\frac{-1\pm\sqrt{1+8}}{2}\]
\[t_{1/2}=\frac{-1\pm3}{2};t_{1}=1,t_{2}=-2\]
\[-2<t<1\]
\[-2<\tg(\frac{x}{2})<1\]
\[\arctg(-2)+k\pi<\frac{x}{2}<\arctg(1)+k\pi\]
\[-\arctg(2)+k\pi<\frac{x}{2}<\frac{\pi}{4}+k\pi\qquad/\cdot2\]
\[-2\arctg(2)+2k\pi<x<\frac{\pi}{2}+2k\pi,k\in\mathbb{Z}\]

% ---------------------------------------------
\subsection{Хомогене неједначине}

Неједначина облика $a\sin^{2}(x)+b\sin(x)\cos(x)+c\cos^{2}(x)\geqslant0$ зовемо хомогеном.

Решавање:

\[a\sin^{2}(x)+b\sin(x)\cos(x)+c\cos^{2}(x)>0\qquad/:\cos^{2}(x)\neq0, x\neq\frac{\pi}{2}+k\pi,k\in\mathbb{Z}\]
\[a\frac{\sin^{2}(x)}{\cos^{2}(x)}+b\frac{\sin(x)\cos(x)}{\cos^{2}(x)}+c\frac{\cos^{2}(x)}{\cos^{2}(x)}>0\]
\[a\tg^{2}(x)+b\tg(x)+c>0\]
\[\text{сменом} \tg(x)=t\]
\[\text{сводимо на квадратну једначину}\]
\[at^{2}+bt+c>0\]

Решавањем ове неједначине и враћањем у смену долазимо до основне неједначине.

% ---------------------------------------------
\subsubsection{Пример 1}

Решити неједначину:

\[2\sin^{2}(x)-5\sin(x)\cos(x)+2\cos^{2}(x)<0\]

Решење:

\[2\sin^{2}(x)-5\sin(x)\cos(x)+2\cos^{2}(x)<0\qquad/:\cos^{2}(x)\]
\[2\frac{\sin^{2}(x)}{\cos^{2}(x)}-\frac{5\sin(x)\cos(x)}{\cos^{2}(x)}+\frac{2\cos^{2}(x)}{\cos^{2}(x)}<0\]
\[2\tg^{2}(x)-5\tg(x)+2<0\]
\[\text{сменом }\tg(x)=t\]
\[2t^{2}-5t+2<0\]
\[t_{1/2}=\frac{5\pm\sqrt{25-16}}{4}\]
\[t_{1/2}=\frac{5\pm3}{4}\]
\[t_{1}=2,t_{2}=\frac{1}{2}\]
\[\frac{1}{2}<t<2\]
\[\frac{1}{2}<\tg(x)<2\]
\[\arctg\left(\frac{1}{2}+k\pi\right)<x<\arctg(2+k\pi),k\in\mathbb{Z}\]

% ---------------------------------------------
\subsection{Линеарна тригонометријска неједначина}

Општи облик: $a\sin(x)+b\cos(x)\gtrless c$

Можемо ову неједначину решавати на два начина.

Први начин: уводимо смену да је

\[\tg\left(\frac{x}{2}\right)=t\]
\[\Rightarrow\sin(x)=\frac{2+\tg\left(\frac{x}{2}\right)}{1+\tg^{2}\left(\frac{x}{2}\right)}=\frac{2t}{1+t^{2}}\]
\[\cos(x)=\frac{1-\tg^{2}\left(\frac{x}{2}\right)}{1+\tg^{2}\left(\frac{x}{2}\right)}=\frac{1-t^{2}}{1+t^{2}}\]

Решен пример 2. у одељку: Неједначине које се сменом своде на основне.

Други начин:

\[a\sin(x)+6\cos(x)>c\qquad/:\sqrt{a^{2}+b^{2}}\]
\[\Rightarrow\frac{a}{\sqrt{a^{2}+b^{2}}}\sin(x)+\frac{b}{\sqrt{a^{2}+b^{2}}}\cos(x)>\frac{c}{\sqrt{a^{2}+b^{2}}}\]

С обзиром да је $\left(\frac{a}{\sqrt{a^{2}+b^{2}}}\right)^{2}+\left(\frac{b}{\sqrt{a^{2}+b^{2}}}\right)^{2}=1$
следи да постоји угао $\varphi$ тако да је

\[\cos(\varphi)=\frac{a}{\sqrt{a^{2}+b^{2}}},\sin(\varphi)=\frac{6}{\sqrt{a^{2}+b^{2}}}\]

или обрнуто враћањем у неједначину

\[\cos(\varphi)\sin(x)+\sin(\varphi)\cos(x)>\frac{c}{\sqrt{a^{2}+b^{2}}}\]
\[\sin(x+\varphi)>\frac{c}{\sqrt{a^{2}+b^{2}}}\]

што се решава као основна.

% ---------------------------------------------
\subsubsection{Пример 1}

Решити неједначину:

\[\sin(x)+\sqrt{3}\cos(x)<-\sqrt{2}\]

Решавање:

\[\sin(x)+\sqrt{3}\cos(x)<-\sqrt{2}\qquad/:\sqrt{1+\sqrt{3}^{2}}=2\]
\[\frac{1}{2}\sin(x)+\frac{\sqrt{3}}{2}\cos(x)<-\frac{\sqrt{2}}{2}\]
\[\cos\left(\frac{\pi}{3}\right)=\frac{1}{2}\quad\text{и}\quad\sin\left(\frac{\pi}{3}\right)=\frac{\sqrt{3}}{2}\]
\[\Rightarrow\cos\left(\frac{\pi}{3}\right)\sin(x)+\sin\left(\frac{\pi}{3}\right)\cos(x)<-\frac{\sqrt{2}}{2}\]
\[\sin\left(x+\frac{\pi}{3}\right)<-\frac{\sqrt{2}}{2}\]

(могли смо да користимо адициону теорему за косинус)

Решење:

\[\frac{5\pi}{4}+2k\pi<x<\frac{7\pi}{4}+2k\pi,\qquad k\in\mathbb{Z}\]
\[x\in\bigcup_{k\in\mathbb{Z}}\left(\frac{5\pi}{4}+2k\pi,\frac{7\pi}{4}+2k\pi\right)\]


% ---------------------------------------------
\section{Разни задаци из неједначина}

\subsection{Задатак 1}

Решити неједначину:

\[\frac{1}{\cos^{2}(x)}-1\geqslant\frac{|\tg(x)-\sqrt{3}|+\sqrt{3}}{\ctg(x)}\]

Решавање:

Zadatak je definisan ako je $\cos(x)\neq0$ и $\ctg(x)\neq0$ $\Rightarrow x\neq\frac{\pi}{2}+k\pi$

\[\frac{1-\cos^{2}(x)}{\cos^{2}(x)}\geqslant\frac{|\tg(x)-\sqrt{3}|+\sqrt{3}}{\ctg(x)}\]
\[\frac{\sin^{2}(x)}{\cos^{2}(x)}\geqslant(|\tg(x)-\sqrt{3}|+\sqrt{3})\cdot\frac{1}{\ctg(x)}\]
\[\tg^{2}(x)\geq(|\tg(x)-\sqrt{3}|+\sqrt{3})\cdot\tg(x)\]
\[\tg(x)[\tg(x)-(|\tg(x)-\sqrt{3}|+\sqrt{3})]\geqslant0\]

Разликујемо два случаја.

Случај 1: za $\tg(x)>\sqrt{3}$

\[\tg(x)[\tg(x)-(\tg(x)-\sqrt{3}+\sqrt{3})]\geqslant0\]
\[\tg(x)\cdot[\tg(x)-\tg(x)]\geqslant0\]
\[\tg(x)\cdot0\geqslant0\]
\[\Rightarrow x\in\mathbb{R}\text{и}x\neq\frac{\pi}{2}+k\pi\]

Решење:

\[\frac{\pi}{3}+k\pi\leqslant x\leqslant\frac{\pi}{2}+k\pi\]
\[R_{1}: x\in\left[\frac{\pi}{3}+k\pi,\frac{\pi}{2}+k\pi\right],k\in\mathbb{Z}\]

Случај 2: za $\tg(x)<\sqrt{3}$

\[\tg^{2}(x)-(-\tg(x)+\sqrt{3}+\sqrt{3})-\tg(x)\geqslant0\]
\[\tg(x)[\tg(x)+\tg(x)-2\sqrt{3}]\geqslant0\]
\[2\tg(x)[\tg(x)-\sqrt{3}]\geqslant0\]
\[\text{пошто је }\tg(x)-\sqrt{3}<0\]
\[2\tg(x)<0\]
\[\tg(x)<0\]

Решење:

\[R_{2}: x=\left(\frac{\pi}{2}+k\pi,\pi+k\pi\right),k\in\mathbb{Z}\]


Решење задатка:

\[R=R_{1}\cup R_{2}\]
\[x\in\left[\frac{\pi}{3}+k\pi,\frac{\pi}{2}+k\pi\right)\cup\left(\frac{\pi}{2}+k\pi,\pi+k\pi\right),k\in\mathbb{Z}\]

% ---------------------------------------------
\subsection{Задатак 2}

Решити неједначину:

\[\arcsin\left(x^{2}-2x-2\right)>\frac{\pi}{4}\]

Решавање:

Дефинисаност једначине:

\[1\leqslant x^{2}-2x-2\leqslant1\]

Случај 1:

\[x^{2}-2x-2\geq-1\]
\[x^{2}-2x-1\geqslant0\]
\[x_{1/2}=\frac{2\pm\sqrt{4+4}}{2}\]
\[x_{1/2}=\frac{2\pm2\sqrt{2}}{2}\]
\[x_{1/2}=1\pm\sqrt{2}\]

Случај 2:

\[x^{2}-2x-2\leq1\]
\[x^{2}-2x-3\leqslant0\]
\[x_{1/2}=\frac{2\pm\sqrt{4+12}}{2}\]
\[x_{1/2}=\frac{2\pm4}{2}\]
\[x_{n}=3,x_{2}=-1\]

Дефинисаност неједначине:

\[x\in[-1,1-\sqrt{2}]\cup[1+\sqrt{2},3]\]

Примена:

\[\arcsin\left(x^{2}-2x-2\right)>\frac{\pi}{4}\]
\[x^{2}-2x-2>\sin\left(\frac{\pi}{4}\right)\]
\[x^{2}-2x-2>\frac{\sqrt{2}}{2}\qquad/\cdot2\]
\[2x^{2}-4x-4>\sqrt{2}\]
\[2x^{2}-4x-4-\sqrt{2}>0\]
\[x_{1/2}=\frac{4\pm\sqrt{16-8(-4-\sqrt{2})}}{4}\]
\[x_{1/2}=\frac{4\pm\sqrt{16+32+8\sqrt{2}}}{4}\]
\[x_{1/2}=\frac{4\pm\sqrt{48+\frac{16\sqrt{2}}{2}}}{4}\]
\[x_{12}=\frac{4\pm4\sqrt{3+\frac{\sqrt{2}}{2}}}{4}\]
\[x_{1+2}=1\pm\sqrt{3+\frac{\sqrt{2}}{2}}\]
\[R:x\in\left[-1,1-\sqrt{3+\frac{\sqrt{2}}{2}}\right)\cup\left(1+\sqrt{3+\frac{\sqrt{2}}{2}},3\right)\]

% ---------------------------------------------
\subsection{Задатак 3}

Решити неједначину:

\[\frac{1}{\tg(x)-1}\leqslant\frac{1}{\tg(x)}\]

Решавање:

Егзистенција задатка:

\[\tg(x)\neq1\text{ и }\tg(x)\neq0\]
\[x\neq\frac{\pi}{4}+k\pi\text{ и }x\neq k\pi\]

\[\frac{1}{\tg(x)-1}-\frac{1}{\tg(x)}\leqslant0\]
\[\frac{1}{\tg(x)-1}-\frac{1}{\tg(x)}\leqslant0\]
\[\frac{1}{\tg(x)(\tg(x)-1)}\leqslant0\]
\[\tg(x)(\tg(x)-1)<0\]
\[[\tg(x)>0\quad\wedge\quad\tg(x)-1<0]\vee[\tg(x)<0\wedge\tg(x)-1>0]\]
\[\tg(x)>0\wedge\tg(x)<1\]
\[0<\tg(x)<1\]
\[x\in\left(k\pi,\frac{\pi}{4}+k\pi\right),k\in\mathbb{Z}\]

% ---------------------------------------------
\newpage

\section{Тригонометријски круг}
    \hspace*{-2em}\begin{tikzpicture}[scale=5,cap=round,>=latex]
        % draw the coordinates
        \draw[->] (-1.5cm,0cm) -- (1.5cm,0cm) node[right,fill=white] {$x$};
        \draw[->] (0cm,-1.5cm) -- (0cm,1.5cm) node[above,fill=white] {$y$};

        % draw the unit circle
        \draw[thick] (0cm,0cm) circle(1cm);

        \foreach \x in {0,30,...,360} {
                % lines from center to point
                \draw[gray] (0cm,0cm) -- (\x:1cm);
                % dots at each point
                \filldraw[black] (\x:1cm) circle(0.4pt);
                % draw each angle in degrees
                \draw (\x:0.6cm) node[fill=white] {$\x^\circ$};
        }

        % draw each angle in radians
        \foreach \x/\xtext in {
            30/\frac{\pi}{6},
            45/\frac{\pi}{4},
            60/\frac{\pi}{3},
            90/\frac{\pi}{2},
            120/\frac{2\pi}{3},
            135/\frac{3\pi}{4},
            150/\frac{5\pi}{6},
            180/\pi,
            210/\frac{7\pi}{6},
            225/\frac{5\pi}{4},
            240/\frac{4\pi}{3},
            270/\frac{3\pi}{2},
            300/\frac{5\pi}{3},
            315/\frac{7\pi}{4},
            330/\frac{11\pi}{6},
            360/2\pi}
                \draw (\x:0.85cm) node[fill=white] {$\xtext$};

        \foreach \x/\xtext/\y in {
            % the coordinates for the first quadrant
            30/\frac{\sqrt{3}}{2}/\frac{1}{2},
            45/\frac{\sqrt{2}}{2}/\frac{\sqrt{2}}{2},
            60/\frac{1}{2}/\frac{\sqrt{3}}{2},
            % the coordinates for the second quadrant
            150/-\frac{\sqrt{3}}{2}/\frac{1}{2},
            135/-\frac{\sqrt{2}}{2}/\frac{\sqrt{2}}{2},
            120/-\frac{1}{2}/\frac{\sqrt{3}}{2},
            % the coordinates for the third quadrant
            210/-\frac{\sqrt{3}}{2}/-\frac{1}{2},
            225/-\frac{\sqrt{2}}{2}/-\frac{\sqrt{2}}{2},
            240/-\frac{1}{2}/-\frac{\sqrt{3}}{2},
            % the coordinates for the fourth quadrant
            330/\frac{\sqrt{3}}{2}/-\frac{1}{2},
            315/\frac{\sqrt{2}}{2}/-\frac{\sqrt{2}}{2},
            300/\frac{1}{2}/-\frac{\sqrt{3}}{2}}
                \draw (\x:1.25cm) node[fill=white] {$\left(\xtext,\y\right)$};

        % draw the horizontal and vertical coordinates
        % the placement is better this way
        \draw (-1.25cm,0cm) node[above=1pt] {$(-1,0)$}
              (1.25cm,0cm)  node[above=1pt] {$(1,0)$}
              (0cm,-1.25cm) node[fill=white] {$(0,-1)$}
              (0cm,1.25cm)  node[fill=white] {$(0,1)$};
    \end{tikzpicture}

% ---------------------------------------------
\newpage

\section{Литература}

\begin{enumerate}
\item Жика Ивановић, Срђан Огњановић: Математика - збирка задатака и тестова за II гимназије и техничких школа
\item Мр. Вене Т. Богославов: Збирка решених задатака из Математике 2
\item Ђ. Дугошија, Ж. Ивановић, Л. Милин: Тригонометрија
\item Јован Д. Кечкић, Стана Ж. Никчеви, Драгана Д. Ранковић: Математика - припрема пријемног испита
\item Јован Д Кечкић: Математика за 2. разред гимназије
\item Општа енциклопедија математике Larousse
\item Бронштејн, Семенђајев, Мусмол, Милинг: Математички приручник
\item Ђорђе Кадијевић: Збирка нестандардних задатака из тригонометрије
\end{enumerate}


\end{document}
