\documentclass[../diplomski.tex]{subfiles}
\begin{document}

\subsection{Неједначина $\sin(x)>a,a\in\mathbb{R}$}

\begin{enumerate}[label=\alph*)]
\item $a<-1$, сваки број је решење тригонометријске неједначине, зато што је $-1\leq\sin(x)\leq1$
\item $a>1$, неједначина нема решење
\item $-1\leq a\leq-1$, приступамо решавању тригонометријске неједначине
\end{enumerate}

Решавање:

\begin{figure}
\centering
\begin{tikzpicture}

  \pic {
    symbol 01={40}{40}{$\arcsin{a}$}{$\pi-\arcsin{a}$}{UP}
  };

\end{tikzpicture}
\caption{A!}
\end{figure}

Решење:

\begin{equation}
\begin{split}
\arcsin(a+2k\pi)<x<\pi-\arcsin(a+2k\pi);k\in\mathbb{Z}\\
x\in\bigcup\limits_{k\in\mathbb{Z}}\arcsin(a+2k\pi);\pi-\arcsin(a+2k\pi)	
\end{split}
\end{equation}

% ---------------------------------------------

\subsubsection{Пример 1}

Решити неједначине:

\begin{enumerate}[label=\alph*)]
\item $\sin(x)>-3$
\item $\sin(x)>3$
\item $\sin(x)>\frac{1}{2}$
\end{enumerate}

Решење:

\begin{enumerate}[label=\alph*)]
\item $\sin(x)>-3$, с обзиром да је $-1\le\sin(x)\le1$ следи да је решење било који реалан број, односно  $x\in\mathbb{R}$
\item $\sin(x)>3$, једначина нема решење због $-1\le\sin(x)\le1$
\item $\sin(x)>\frac{1}{2}$
\end{enumerate}

\begin{figure}[!ht]
\centering
\vspace{110pt}%
\begin{tikzpicture}[transform canvas={scale=0.6}]

  \tikzstyle{every node}=[font=\fontsize{20}{20}\selectfont]

  \pic[yshift=20pt] (PicB-) {
    symbol 01=
      {30}{30}
      {$\arcsin{\frac{1}{2}}=\frac{\pi}{2}$}
      {$\pi-\arcsin{\frac{1}{2}=\frac{5\pi}{6}}$}
      {UP}{OUT}
	};
		
	\draw[thick] (PicB-O) -- node[right] {$\frac{1}{2}$} ++ (PicB-x0yR2);

\end{tikzpicture}
\caption{}
\end{figure}


Решење:

\[\frac{\pi}{6}+2k\pi<x<\frac{5\pi}{6}+2k\pi,k\in\mathbb{Z}\]
\centerline{или}
\[x\in\bigcup_{k\in\mathbb{Z}}\frac{\pi}{6}+2k\pi,\frac{5\pi}{6}+2k\pi\]

% ---------------------------------------------

\subsubsection{Пример 2}

Решити неједначину: $2\sin(2x-\frac{\pi}{3})=1$

Решење:

\[2\sin(2x-\frac{\pi}{3})=1\]

ако решавамо по формули

\[\arcsin\left(\frac{1}{2}+2k\pi\right)<2x-\frac{\pi}{3}<\pi-\arcsin\left(\frac{1}{2}+2k\pi\right)\]
\[\frac{\pi}{6}+2k\pi<2x-\frac{\pi}{3}<\pi-\frac{\pi}{6}+2k\pi\]
\[\frac{\pi}{6}+2k\pi<2x-\frac{\pi}{3}<\frac{5\pi}{6}+2k\pi\]
\[\frac{\pi}{3}+\frac{\pi}{6}+2k\pi<2x<\frac{\pi}{3}+\frac{5\pi}{6}+2k\pi\]
\[\frac{\pi}{2}+2k\pi<2x<\frac{7\pi}{6}+2k\pi\]
\[\frac{\pi}{4}+k\pi<x<\frac{7\pi}{12}+k\pi,k\in\mathbb{R}\]
\centerline{или}
\[x\in\bigcup_{k\in\mathbb{Z}}(\frac{\pi}{4}+k\pi,\frac{7\pi}{12}+k\pi)\]




\end{document}