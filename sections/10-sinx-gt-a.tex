\documentclass[../diplomski.tex]{subfiles}
\begin{document}

\subsection{Неједначина $\sin(x)>a,a\in\mathbb{R}$}

\begin{enumerate}[label=\alph*)]
\item $a<-1$, сваки број је решење тригонометријске неједначине, зато што је $-1\leq\sin(x)\leq1$
\item $a>1$, неједначина нема решење
\item $-1\leq a\leq-1$, приступамо решавању тригонометријске неједначине
\end{enumerate}

Решавање:

\subfile{figures/fig-A.tex}

Решење:

\begin{equation}
\begin{split}
\arcsin(a+2k\pi)<x<\pi-\arcsin(a+2k\pi);k\in\mathbb{Z}\\
x\in\bigcup\limits_{k\in\mathbb{Z}}\arcsin(a+2k\pi);\pi-\arcsin(a+2k\pi)	
\end{split}
\end{equation}

% ---------------------------------------------

\subsubsection{Пример 1}

Решити неједначине:

\begin{enumerate}[label=\alph*)]
\item $\sin(x)>-3$
\item $\sin(x)>3$
\item $\sin(x)>\frac{1}{2}$
\end{enumerate}

Решење:

\begin{enumerate}[label=\alph*)]
\item $\sin(x)>-3$, с обзиром да је $-1\le\sin(x)\le1$ следи да је решење било који реалан број, односно  $x\in\mathbb{R}$
\item $\sin(x)>3$, једначина нема решење због $-1\le\sin(x)\le1$
\item $\sin(x)>\frac{1}{2}$
\end{enumerate}


\begin{figure}
\centering
	\begin{tikzpicture}
	\clip
	  (-5.5,5.5) rectangle (5.5,-0.5);
	\draw[->]
	  (-5.5,0) -- (5,0) node[right] {$x$};
	\draw[->]
	  (0,-0.5) -- (0,5) node[above] {$y$};
	\coordinate (O) at (0,0);
	\coordinate (aux2) at (-4,{4*tan(40)});
	\coordinate (aux3) at (4,{4*tan(40)});
	\coordinate (left) at (-{4*cos(40)},{4*sin(40)});
	\coordinate (right) at ({4*cos(40)},{4*sin(40)});
	\coordinate (leftUP) at (-{4*cos(40)},5);
	\coordinate (rightUP) at ({4*cos(40)},5);
	\coordinate (leftArc) at (-{4.5*cos(40)},{4.5*sin(40)});
	\coordinate (rightArc) at ({4.5*cos(40)},{4.5*sin(40)});
	\draw[->]
	  (O) -- (aux3) node[above] {$\arcsin{\frac{1}{2}}=\frac{\pi}{2}$};
	\draw[->]
	  (O) -- (aux2);
	\draw
	  (left) -- (right);
	\draw[thick,red!70!black] 
	  (O) circle (4);
	\draw[red,<->] % arc
		(O) ++ (leftArc) arc(140:40:4.5);
	\path[clip] (O) circle (4); % fill
	\fill[draw=black, thick, pattern=north west lines]
		(left) -- (leftUP) -- (rightUP) -- (right) -- cycle;
	\end{tikzpicture}
\caption{}
\end{figure}

Решење:

\[\frac{\pi}{6}+2k\pi<x<\frac{5\pi}{6}+2k\pi,k\in\mathbb{Z}\]
\centerline{или}
\[x\in\bigcup_{k\in\mathbb{Z}}\frac{\pi}{6}+2k\pi,\frac{5\pi}{6}+2k\pi\]

% ---------------------------------------------

\subsubsection{Пример 2}

Решити неједначину: $2\sin(2x-\frac{\pi}{3})=1$

Решење:

\[2\sin(2x-\frac{\pi}{3})=1\]

ако решавамо по формули

\[\arcsin\left(\frac{1}{2}+2k\pi\right)<2x-\frac{\pi}{3}<\pi-\arcsin\left(\frac{1}{2}+2k\pi\right)\]
\[\frac{\pi}{6}+2k\pi<2x-\frac{\pi}{3}<\pi-\frac{\pi}{6}+2k\pi\]
\[\frac{\pi}{6}+2k\pi<2x-\frac{\pi}{3}<\frac{5\pi}{6}+2k\pi\]
\[\frac{\pi}{3}+\frac{\pi}{6}+2k\pi<2x<\frac{\pi}{3}+\frac{5\pi}{6}+2k\pi\]
\[\frac{\pi}{2}+2k\pi<2x<\frac{7\pi}{6}+2k\pi\]
\[\frac{\pi}{4}+k\pi<x<\frac{7\pi}{12}+k\pi,k\in\mathbb{R}\]
\centerline{или}
\[x\in\bigcup_{k\in\mathbb{Z}}(\frac{\pi}{4}+k\pi,\frac{7\pi}{12}+k\pi)\]




\end{document}