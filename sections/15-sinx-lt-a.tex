\documentclass[../diplomski.tex]{subfiles}
\begin{document}

\subsection{Неједначина $\sin(x)<a,a\in\mathbb{R}$}

\begin{enumerate}[label=\alph*)]
\item $a<-1$, с обзиром да је $-1\leq\sin(x)\leq1$ у овом случају немамо решење, односно: $x=\O$
\item $a>1$, сваки реалан број је решење, односно $x\in\mathbb{R}$
\item $-1\leq a\leq-1$, приступамо решавању тригонометријске неједначине
\end{enumerate}

Решавање:

\begin{figure}[!ht]
\centering
\vspace{180pt}%
\begin{tikzpicture}[transform canvas={scale=0.6}]

  \tikzstyle{every node}=[font=\fontsize{20}{20}\selectfont]
  
  \pic[yshift=140pt] {
    symbol 01=
      {145}{-215}
      {$\pi-arcsin{a}$}
      {$2\pi+arcsin{a}$}
      {FULL}
      {IN}
  };

\end{tikzpicture}
\caption{}
\end{figure}


\begin{equation}
\begin{split}
\pi-\arcsin(a+2k\pi)<x<2\pi+\arcsin(a+2k\pi);k\in\mathbb{Z}\\
x\in\bigcup\limits_{k\in\mathbb{Z}}(\pi-\arcsin(a+2k\pi);2\pi+\arcsin(a+2k\pi))
\end{split}
\end{equation}


% ---------------------------------------------
\subsubsection{Пример 1}

Решити неједначине:

\begin{enumerate}[label=\alph*)]
\item $\sin(x)<-2$, једначина нема решење
\item $\sin(x)<2$, решење: $x\in\mathbb{R}$
\item $\sin(x)<-\frac{\sqrt3}{2}$
\end{enumerate}

\begin{figure}
\centering
\begin{tikzpicture}

  \pic {
  	symbol 01={225}{-135}
  		{$\pi-\arcsin(\frac{-\sqrt(3)}{-2})=\frac{4\pi}{3}$}
  		{$2\pi+arcsin{\frac{-\sqrt(3)}{2}}=\frac{5\pi}{3}$}
  		{DOWN}
  };

\end{tikzpicture}
\caption{D!}
\end{figure}


\[\frac{4\pi}{3}+2x\pi<x<\frac{5\pi}{3}+2k\pi,x\in\mathbb{Z}\]
\[x\in\bigcup_{k\in\mathbb{Z}}(\frac{4\pi}{3}+2k\pi,\frac{5\pi}{3}+2k\pi)\]

ако решавамо по формули

\[\pi-\arcsin\left(-\frac{\sqrt{3}}{2}\right)+2k\pi<x<2\pi+\arcsin\left(-\frac{\sqrt{3}}{2}\right)+2k\pi\]
\[\pi+\frac{\pi}{3}+2k\pi<x<2\pi-\frac{\pi}{3}+2k\pi\]
\[\frac{4\pi}{3}+2k\pi<x<\frac{5\pi}{3}+2k\pi\]
\centerline{или}
\[x\in\bigcup_{k\in\mathbb{Z}}\left(\frac{4\pi}{3}+2k\pi,\frac{5\pi}{3}+2k\pi\right)\]


% ---------------------------------------------
\subsubsection{Пример 2}

Решити неједначину:

\[\sin(x)\cos\left(\frac{\pi}{6}\right)+\cos(x)\sin\left(\frac{\pi}{6}\right)<\frac{1}{2}\]

Решење:

\begin{figure}
\centering
	\begin{tikzpicture}
	\clip
	  (-5.5,5.5) rectangle (5.5,-5.5);
	\end{tikzpicture}
\caption{E!}
\end{figure}


\[\sin(x)\cos\left(\frac{\pi}{6}\right)+\cos(x)\sin\left(\frac{\pi}{6}\right)<\frac{1}{2}\]
\[\sin\left(x+\frac{\pi}{6}\right)<\frac{1}{2}\]
\[\frac{5\pi}{6}+2k\pi<x+\frac{\pi}{6}<\frac{13\pi}{6}+2k\pi\]
\[\frac{5\pi}{6}-\frac{\pi}{6}+2k\pi<x<\frac{13\pi}{6}-\frac{\pi}{6}+2k\pi\]
\[\frac{4\pi}{6}+2k\pi<x<\frac{12\pi}{6}+2k\pi\]
\[\frac{2\pi}{3}+2k\pi<x<2\pi+2k\pi\]
\centerline{или}
\[x\in\bigcup_{k\in\mathbb{Z}}\left(\frac{2\pi}{3}+2k\pi,2\pi+2k\pi\right)\]

Могли смо да решимо користећи формулу XX


% ---------------------------------------------
\subsubsection{Пример 3}

Решити неједначину: $\sin(x)+\cos(x)<\sqrt{2}$

Решење:

\[\sin(x)+\cos(x)<\sqrt{2}\qquad/:\sqrt{2}\]
\[\frac{1}{\sqrt{2}}\sin(x)+\frac{1}{\sqrt{2}}\cos(x)<\frac{\sqrt{2}}{\sqrt{2}}\]
\[\frac{\sqrt{2}}{2}\sin(x)+\frac{\sqrt{2}}{2}\cos(x)<1\]
\[(\frac{\sqrt{2}}{2}=\cos(\frac{\pi}{4});\frac{\sqrt{2}}{2}=\sin(\frac{\pi}{4}))\]
\[\cos(\frac{\pi}{4})\sin(x)+\sin(\frac{\pi}{4})\cos(x)<1\]
\[\sin(x+\frac{\pi}{4})<1\qquad\text{(на основу адиционе теореме)}\]
\[\text{Пошто је} -1\leqslant\sin(x)\leqslant1\Rightarrow \text{решење је} \forall x\in\mathbb{R}\]
\[\text{сем } \sin\left(x+\frac{\pi}{4}\right)=1\text{ односно }x+\frac{\pi}{4}=\frac{\pi}{2}+2k\pi,k\in\mathbb{Z}\]
\centerline{даје решење}
\[\forall x,x\neq\frac{\pi}{4}+2x\pi,k\in\mathbb{Z}\]



\end{document}