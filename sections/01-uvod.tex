\documentclass[../diplomski.tex]{subfiles}
\begin{document}


Тригонометрија је део математике и геометрије која се бави израчунавањем
елемената троугла, проналажењем зависности у њиховим односима, као и
успостављањем функција углова који их дефинишу.

Првобитно је искључиво израчунавана вредност елемената троугла.

Њен првобитни циљ је данас превазиђен и примена тригонометрије на основу
израчунавања тригонометријских функција, ван сваког посматрања троугла,
учинило је од тригонометрије значајну област математике и геометрије.
Примењује се у различитим областима као што су инжењерство, архитектура,
геодезија, навигација и астрономија.

Особа одговорна за "модерну" тригонометрију био је ренесансни
математичар из Немачке \hl{Јохан Милер фин Кенигсберг} - познатији као
\hl{Регмомонтанус}. Био је први који је тригонометрију третирао као субјекат
по себи. Даљи напредак тригонометрије направио је Никола Коперник.

\end{document}