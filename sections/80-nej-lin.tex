\documentclass[../diplomski.tex]{subfiles}
\begin{document}

\subsection{Линеарна тригонометријска неједначина}

Општи облик: $a\sin(x)+b\cos(x)\gtrless c$

Можемо ову неједначину решавати на два начина.

Први начин: уводимо смену да је

\[\tg\left(\frac{x}{2}\right)=t\]
\[\Rightarrow\sin(x)=\frac{2+\tg\left(\frac{x}{2}\right)}{1+\tg^{2}\left(\frac{x}{2}\right)}=\frac{2t}{1+t^{2}}\]
\[\cos(x)=\frac{1-\tg^{2}\left(\frac{x}{2}\right)}{1+\tg^{2}\left(\frac{x}{2}\right)}=\frac{1-t^{2}}{1+t^{2}}\]

Решен пример 2. у одељку: Неједначине које се сменом своде на основне.

Други начин:

\[a\sin(x)+6\cos(x)>c\qquad/:\sqrt{a^{2}+b^{2}}\]
\[\Rightarrow\frac{a}{\sqrt{a^{2}+b^{2}}}\sin(x)+\frac{b}{\sqrt{a^{2}+b^{2}}}\cos(x)>\frac{c}{\sqrt{a^{2}+b^{2}}}\]

С обзиром да је $\left(\frac{a}{\sqrt{a^{2}+b^{2}}}\right)^{2}+\left(\frac{b}{\sqrt{a^{2}+b^{2}}}\right)^{2}=1$
следи да постоји угао $\varphi$ тако да је

\[\cos(\varphi)=\frac{a}{\sqrt{a^{2}+b^{2}}},\sin(\varphi)=\frac{6}{\sqrt{a^{2}+b^{2}}}\]

или обрнуто враћањем у неједначину

\[\cos(\varphi)\sin(x)+\sin(\varphi)\cos(x)>\frac{c}{\sqrt{a^{2}+b^{2}}}\]
\[\sin(x+\varphi)>\frac{c}{\sqrt{a^{2}+b^{2}}}\]

што се решава као основна.

% ---------------------------------------------
\subsubsection{Пример 1}

Решити неједначину:

\[\sin(x)+\sqrt{3}\cos(x)<-\sqrt{2}\]

Решавање:

\[\sin(x)+\sqrt{3}\cos(x)<-\sqrt{2}\qquad/:\sqrt{1+\sqrt{3}^{2}}=2\]
\[\frac{1}{2}\sin(x)+\frac{\sqrt{3}}{2}\cos(x)<-\frac{\sqrt{2}}{2}\]
\[\cos\left(\frac{\pi}{3}\right)=\frac{1}{2}\quad\text{и}\quad\sin\left(\frac{\pi}{3}\right)=\frac{\sqrt{3}}{2}\]
\[\Rightarrow\cos\left(\frac{\pi}{3}\right)\sin(x)+\sin\left(\frac{\pi}{3}\right)\cos(x)<-\frac{\sqrt{2}}{2}\]
\[\sin\left(x+\frac{\pi}{3}\right)<-\frac{\sqrt{2}}{2}\]

(могли смо да користимо адициону теорему за косинус)

Решење:

\begin{figure}[!ht]
\centering
\vspace{100pt}%
\begin{tikzpicture}[transform canvas={scale=0.6}]

  \tikzstyle{every node}=[font=\fontsize{20}{20}\selectfont]

  \pic[yshift=140pt] {
    symbol 01=
      {225}{-135}
      {$\pi+\frac{\pi}{4}=\frac{5\pi}{4}$}
      {$2\pi-\frac{\pi}{4}=\frac{7\pi}{4}$}
      {DOWN}
      {OUT}
  };

\end{tikzpicture}
\caption{}
\end{figure}


\[\frac{5\pi}{4}+2k\pi<x<\frac{7\pi}{4}+2k\pi,\qquad k\in\mathbb{Z}\]
\[x\in\bigcup_{k\in\mathbb{Z}}\left(\frac{5\pi}{4}+2k\pi,\frac{7\pi}{4}+2k\pi\right)\]


\end{document}