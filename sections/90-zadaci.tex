\documentclass[../diplomski.tex]{subfiles}
\begin{document}

% ---------------------------------------------
\subsection{Задатак 1}

Решити неједначину:

\[\frac{1}{\cos^{2}(x)}-1\geqslant\frac{|\tg(x)-\sqrt{3}|+\sqrt{3}}{\ctg(x)}\]

Решавање:

Zadatak je definisan ako je $\cos(x)\neq0$ и $\ctg(x)\neq0$ $\Rightarrow x\neq\frac{\pi}{2}+k\pi$

\[\frac{1-\cos^{2}(x)}{\cos^{2}(x)}\geqslant\frac{|\tg(x)-\sqrt{3}|+\sqrt{3}}{\ctg(x)}\]
\[\frac{\sin^{2}(x)}{\cos^{2}(x)}\geqslant(|\tg(x)-\sqrt{3}|+\sqrt{3})\cdot\frac{1}{\ctg(x)}\]
\[\tg^{2}(x)\geq(|\tg(x)-\sqrt{3}|+\sqrt{3})\cdot\tg(x)\]
\[\tg(x)[\tg(x)-(|\tg(x)-\sqrt{3}|+\sqrt{3})]\geqslant0\]

Разликујемо два случаја.

Случај 1: za $\tg(x)>\sqrt{3}$

\[\tg(x)[\tg(x)-(\tg(x)-\sqrt{3}+\sqrt{3})]\geqslant0\]
\[\tg(x)\cdot[\tg(x)-\tg(x)]\geqslant0\]
\[\tg(x)\cdot0\geqslant0\]
\[\Rightarrow x\in\mathbb{R}\text{и}x\neq\frac{\pi}{2}+k\pi\]

Решење:

\[\frac{\pi}{3}+k\pi\leqslant x\leqslant\frac{\pi}{2}+k\pi\]
\[R_{1}: x\in\left[\frac{\pi}{3}+k\pi,\frac{\pi}{2}+k\pi\right],k\in\mathbb{Z}\]

Случај 2: za $\tg(x)<\sqrt{3}$

\[\tg^{2}(x)-(-\tg(x)+\sqrt{3}+\sqrt{3})-\tg(x)\geqslant0\]
\[\tg(x)[\tg(x)+\tg(x)-2\sqrt{3}]\geqslant0\]
\[2\tg(x)[\tg(x)-\sqrt{3}]\geqslant0\]
\[\text{пошто је }\tg(x)-\sqrt{3}<0\]
\[2\tg(x)<0\]
\[\tg(x)<0\]

Решење:

\[R_{2}: x=\left(\frac{\pi}{2}+k\pi,\pi+k\pi\right),k\in\mathbb{Z}\]


Решење задатка:

\[R=R_{1}\cup R_{2}\]
\[x\in\left[\frac{\pi}{3}+k\pi,\frac{\pi}{2}+k\pi\right)\cup\left(\frac{\pi}{2}+k\pi,\pi+k\pi\right),k\in\mathbb{Z}\]

% ---------------------------------------------
\subsection{Задатак 2}

Решити неједначину:

\[\arcsin\left(x^{2}-2x-2\right)>\frac{\pi}{4}\]

Решавање:

Дефинисаност једначине:

\[1\leqslant x^{2}-2x-2\leqslant1\]

Случај 1:

\[x^{2}-2x-2\geq-1\]
\[x^{2}-2x-1\geqslant0\]
\[x_{1/2}=\frac{2\pm\sqrt{4+4}}{2}\]
\[x_{1/2}=\frac{2\pm2\sqrt{2}}{2}\]
\[x_{1/2}=1\pm\sqrt{2}\]

Случај 2:

\[x^{2}-2x-2\leq1\]
\[x^{2}-2x-3\leqslant0\]
\[x_{1/2}=\frac{2\pm\sqrt{4+12}}{2}\]
\[x_{1/2}=\frac{2\pm4}{2}\]
\[x_{n}=3,x_{2}=-1\]

Дефинисаност неједначине:

\[x\in[-1,1-\sqrt{2}]\cup[1+\sqrt{2},3]\]

Примена:

\[\arcsin\left(x^{2}-2x-2\right)>\frac{\pi}{4}\]
\[x^{2}-2x-2>\sin\left(\frac{\pi}{4}\right)\]
\[x^{2}-2x-2>\frac{\sqrt{2}}{2}\qquad/\cdot2\]
\[2x^{2}-4x-4>\sqrt{2}\]
\[2x^{2}-4x-4-\sqrt{2}>0\]
\[x_{1/2}=\frac{4\pm\sqrt{16-8(-4-\sqrt{2})}}{4}\]
\[x_{1/2}=\frac{4\pm\sqrt{16+32+8\sqrt{2}}}{4}\]
\[x_{1/2}=\frac{4\pm\sqrt{48+\frac{16\sqrt{2}}{2}}}{4}\]
\[x_{12}=\frac{4\pm4\sqrt{3+\frac{\sqrt{2}}{2}}}{4}\]
\[x_{1+2}=1\pm\sqrt{3+\frac{\sqrt{2}}{2}}\]
\[R:x\in\left[-1,1-\sqrt{3+\frac{\sqrt{2}}{2}}\right)\cup\left(1+\sqrt{3+\frac{\sqrt{2}}{2}},3\right)\]

% ---------------------------------------------
\subsection{Задатак 3}

Решити неједначину:

\[\frac{1}{\tg(x)-1}\leqslant\frac{1}{\tg(x)}\]

Решавање:

Егзистенција задатка:

\[\tg(x)\neq1\text{ и }\tg(x)\neq0\]
\[x\neq\frac{\pi}{4}+k\pi\text{ и }x\neq k\pi\]

\[\frac{1}{\tg(x)-1}-\frac{1}{\tg(x)}\leqslant0\]
\[\frac{1}{\tg(x)-1}-\frac{1}{\tg(x)}\leqslant0\]
\[\frac{1}{\tg(x)(\tg(x)-1)}\leqslant0\]
\[\tg(x)(\tg(x)-1)<0\]
\[[\tg(x)>0\quad\wedge\quad\tg(x)-1<0]\vee[\tg(x)<0\wedge\tg(x)-1>0]\]
\[\tg(x)>0\wedge\tg(x)<1\]
\[0<\tg(x)<1\]
\[x\in\left(k\pi,\frac{\pi}{4}+k\pi\right),k\in\mathbb{Z}\]


\end{document}